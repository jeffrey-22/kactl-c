\chapter{Various}

\section{Intervals}
	\kactlimport{IntervalContainer.h}
	\kactlimport{IntervalCover.h}
	\kactlimport{ConstantIntervals.h}

\section{Misc. algorithms}
	\kactlimport{TernarySearch.h}
	\kactlimport{LIS.h}

\section{Dynamic programming}
	\kactlimport{KnuthDP.h}
	\kactlimport{DivideAndConquerDP.h}

\section{Debugging tricks}
	\begin{itemize}
		\item \verb@signal(SIGSEGV, [](int) { _Exit(0); });@ converts segfaults into Wrong Answers.
			Similarly one can catch SIGABRT (assertion failures) and SIGFPE (zero divisions).
			\verb@_GLIBCXX_DEBUG@ failures generate SIGABRT (or SIGSEGV on gcc 5.4.0 apparently).
		\item \verb@feenableexcept(29);@ kills the program on NaNs (\texttt 1), 0-divs (\texttt 4), infinities (\texttt 8) and denormals (\texttt{16}).
	\end{itemize}

\section{Optimization tricks}
	\verb@__builtin_ia32_ldmxcsr(40896);@ disables denormals (which make floats 20x slower near their minimum value).
	\subsection{Bit hacks}
		\begin{itemize}
			\item \verb@x & -x@ is the least bit in \texttt{x}.
			\item \verb@for (int x = m; x; ) { --x &= m; ... }@ loops over all subset masks of \texttt{m} (except \texttt{m} itself).
			\item \verb@c = x&-x, r = x+c; (((r^x) >> 2)/c) | r@ is the next number after \texttt{x} with the same number of bits set.
			\item \verb@rep(b,0,K) rep(i,0,(1 << K))@ \\ \verb@  if (i & 1 << b) D[i] += D[i^(1 << b)];@ computes all sums of subsets.
		\end{itemize}
	\subsection{Pragmas}
		\begin{itemize}
			\item \lstinline{#pragma GCC optimize ("Ofast")} will make GCC auto-vectorize loops and optimizes floating points better.
			\item \lstinline{#pragma GCC target ("avx2")} can double performance of vectorized code, but causes crashes on old machines.
			\item \lstinline{#pragma GCC optimize ("trapv")} kills the program on integer overflows (but is really slow).
		\end{itemize}
	\kactlimport{FastMod.h}
	\kactlimport{FastInput.h}
	% \kactlimport{BumpAllocator.h}
	% \kactlimport{SmallPtr.h}
	% \kactlimport{BumpAllocatorSTL.h}
	% \kactlimport{Unrolling.h}
	% \kactlimport{SIMD.h}
	% \kactlimport{Idk.h}
	\kactlimport{sos.h}
	\kactlimport{XorBasis.h}
	\kactlimport{Dijkstra.h}
	\kactlimport{nkdp.h}
\section{Techniques}

\subsection{$O(nk)$ Tree DP}
Assume you have $n$ rocks with nonnegative integer weights $a_1,a_2,\cdots,a_n$ such that $a_1+a_2+\cdots+a_n=m$
You want to find out if there is a way to choose some rocks such that their total weight is $w$/
Suppose there are three rocks with equal weights $a,a,a$/
Notice that it doesn't make any difference if we replace these three rocks with two rocks with weights $a,2a$/
We can repeat this process of replacing until there are at most two rocks of each weight. The sum of weights is still $m$,
so there can be only $O(\sqrt{m})$ rocks. Now you can use a classical DP algorithm but with only $O(\sqrt{m})$
elements, which can be lead to a better complexity in many cases.
\end{itemize}
This trick mostly comes up when the $a_1,a_2,\cdots,a_n$ form a partition of some kind. 
For example, maybe they represent connected components of a graph.
\subsection{Slope Trick DP}
Slope-trick-able: Continuous, sections with integer slope, convex/concave.
Store the function with the rightmost section and a multiset of all the slope changing points where the slope changes its value by 1.
Sum of slope-trick-able functions is slope-trick-able.
Prefix min of slope-trick-able functions is slope-trick-able. Or local minimum.
\subsection{Knapsack and subset sum}
There are $N$ items, with weight $w_i$. Say $C=\sum w_i$. For each $k\in [1,C]$, can we find set $S$ such that $\sum_{i\in S}w_i=k$?
Solvable in $O(\frac{C\sqrt{C}}{32})$: If $w_i$ repeats for $occ_i$ times, decompose into powers of $2$.
$occ_i=12 \to \{w_i, 2w_i, 4w_i, 5w_i\}$. Number of $(w_i, occ_i)$ tuples is only $\sqrt{C}$.
Alternatively: process weights from small to large. Say smallest weight is $w$ with $occ$ occurrences. 
We have the current list as well as an initially empty new list as the final new weights.
If $occ$ is odd, add $(2w, \frac{occ-1}{2})$ to the current list and $(w, 1)$ to a new list. 
For even add $(2w, \frac{occ-2}{2})$ to the current list and $(w, 2)$ to the new list.
Now we have $\sqrt{C}$ items, so just dp with an array of $C$ $0/1$ bits.
\subsection{Bipartite matching with one pair removed}
Let's build the perfect matching for initial graph $M$. When we remove vertices $u$ and $v$, let's remove the
affected edges from $M$, get the matching $M'$. At most two edges will be removed, so we only need to
run one phase of Kuhn's algorithm to build $M_{uv}$ from $M'$ (or find out that it is impossible). Now the
complexity is $O(n^2 m)$ which is better but still too slow. Let's improve it even more.
Imagine if we remove verices $v$ and $u$. If the edge $uv$ is in $M$, then $M'$
is a perfect matching. If not, then we have two unsaturated vertices in $M'$, let's say $v'$ and $u'$. 
We can build the perfect matching $M_{uv}$ iff there is an alternating path 
(An alternating path is a path in which the edges alternately belong / do not belong to the matching.)
from $v'$ to $u'$. 
We can precalculate for each pair of nodes $(v', u')$ if there is an
altertating path from $v'$ to $u'$ in $O(nm)$ simply by running DFS from each node $v'$ of the left part.
Now we can check for each pair $(u, v)$ if it is good or not in $O(1)$. 
So the total complexity will be $O(nm + n^2)$.